\section{Vorbereitung}
In diesem Versuch Analysieren wir das Impulshöhenspektrum der $ \gamma $-Strahlung von Cs-137, Na-22 und Co-60. Zusätzlich ermitteln wir das bestehende Untergrundspektrum während keins der Präparate genutzt wird. Das somit erhaltene Spektrum wird in den Aufgaben 1 bis 3 von den gemessenen Werten abgezogen.
\begin{figure}
	\centering
	\includegraphics[width=\textwidth]{../Daten/Aufgabe 1/Untergrund.png}
	\caption{Untergrundspektrum des Versuchs}
\end{figure}
Nach unserem ersten gemessenen Präparat konnten wir zudem das Verhältnis von Kanalanzahl zu Energie errechnen, welches 1 Kanal = 1,32863 keV beträgt.
\section{Cs-137}
Als erstes untersuchen wir das Präparat Cs-137. Um die Position von Photopeaks, Röntgenpeaks und Rückstreupeaks zu ermitteln nutzen wir die Funktion von CASSY die diese ausgibt. Die Position der Comptonkante schätzen wir.
\begin{figure}
	\centering
	\includegraphics[width=\textwidth]{../Daten/Aufgabe 1/1_CS_2.png}
	\caption{Impulshöhenspektrum des Cs-137 Präparats}
\end{figure}
Der Röntgenpeak befindet sich bei 43,952 keV, der Rückstreupeak bei 212,767 keV, die Comptonkante ist bei 458,257 keV und der Photopeak bei 662 keV. Ein Vergleich mit dem gegebenen Literaturwert ergibt hier keinen Sinn, da diese Messung zur Eichung der Skala genutzt wurde.
\section{Co-60}
Als nächstes untersuchen wir das Präparat Co-60.
\begin{figure}
	\centering
	\includegraphics[width=\textwidth]{../Daten/Aufgabe 1/1_CO_2.png}
	\caption{Impulshöhenspektrum des Cs-137 Präparats}
\end{figure}
Der erste Rückstreupeak befindet sich bei 154,816 keV, der zweite bei 244,229 keV, die Comptonkante ist bei 897,181 keV, der erste Photopeak ist bei 1114,81 keV und der zweite bei 1251,28 keV. Vergleicht man diese mit den gegebenen Literaturwerten erhält man für den Photopeak (E$ _{Lit,1} $=1173 keV) einen Fehler von 4,96\%. Für den zweiten (E$ _{Lit,2} $=1333 keV) ergibt sich ein Fehler von 6,13\%.
\section{Na-22}
Als letztes untersuchen wir das Präparat Na-22.
\begin{figure}
	\centering
	\includegraphics[width=\textwidth]{../Daten/Aufgabe 1/1_NA_2.png}
	\caption{Impulshöhenspektrum des Na-22 Präparats}
\end{figure}
Der erste Rückstreupeak befindet sich bei 50,134 keV, der zweite bei 204,627 keV, die erste Comptonkante ist bei 332,352 keV und die zweite bei 972,505 keV, der erste Photopeak ist bei 519,732 keV und der zweite bei 1192,77 keV. Vergleicht man diese mit den gegebenen Literaturwerten erhält man für den Photopeak (E$ _{Lit,1} $=511 keV) einen Fehler von 1,68\%. Für den zweiten (E$ _{Lit,2} $=1275 keV) ergibt sich ein Fehler von 6,45\%. 
\section{Auswertung}
Zusammenfassend lässt sich sagen, dass für höhere Energien die Werte stärker von den Literaturwerten abweichen als für niedrige, was wohl auch daran liegt, dass die Eichung bei einem eher niedrigem Energieniveau durchgeführt wurde. Es ist auch sichtbar, dass die Rückstreupeaks ungefähr die Energie des zugehörigen Photopeaks minus die Energie der Comptonkante besitzen. Die Comptonkante ist Material-abhängig und über\\ $ ( 1-1/1+\frac{2E_{Max}}{m_0c^2})  $ berechnen, wobei $ E_{Max} $ die Maximalenergie des Elektrons ist. Zudem ist sichtbar, dass nur beim Cs Präparat der Röntgenpeak aufgenommen wird.
\section{Versuchsaufbau}
In diesem Versuchsteil wird nun effektiv der gleiche Versuch wie in Kapitel 1 durchgefuehrt, mit dem entscheidenden Unterschied, dass anstelle des verwendeten Hg nun Neon zum Einsatz kommt.
Da ein anderes Gas verwendet wird, werden auch andere Anregungsenergien erwartet.
Bei der Anregung von Neon durch Elektronen ist ein Uebergang von Grundzustand in einen der 10 3p Zustaende am wahrscheinlichsten. Diese Liegen in einem Bereich von 18,4 - 19 V \cite{HND1}.
Beim Uebergang vom 3p in den 3s zustand werden Photonen im Sichtbaren bereich emittiert. Der Uebergang aus dem 3s zustand in den Grundzustand emmitiert Photonen im UV bereich.
\subsection{Heizen des Versuchsaufbaus}
Heizen des Versuchsaufbaus ergibt hier wenig Sinn, da nun ein Gas anstelle eines Dampfes vorliegt. Damit aendert sich die Teilchendichte nicht mit der Temperatur.
\section{Beobachtungen und Interpretation}
Die in \ref{tab:gemesseneAnregungsenergieNeon} gelisteten, gemessenen Maxima bestaetigen die geaeusserten Vermutungen.
Im Stossraum koennen zudem pro gemessenen Peak eine leuchtende Scheibe gesehen werden. Diese Scheiben sind die Stosszonen der verschiedenen Elektronen.
\begin{table}
	\centering
	\caption{Anregungsniveaus von Neon}
	\begin{tabular}{c c}
		Anzahl der Stoesse & Gegenspannung des Peaks in V \\
		\midrule
		1 & 18,3 \\
		2 & 25,6 \\
		3 & 55,7 \\
	\end{tabular}
	\label{tab:gemesseneAnregungsenergieNeon}
\end{table}
Eine Lineare Regression der gemessenen werte ergibt eine Steigung von 18,7 V / Anregung was Genau im von der Vorbereitungsliteratur angegebenen Intervalls liegt.


\section{Versuchsaufbau}
In diesem Versuchsteil wird nun effektiv der gleiche Versuch wie in Kapitel 1 durchgeführt, mit dem entscheidenden Unterschied, dass anstelle des verwendeten Quecksilbers nun Neon zum Einsatz kommt.
Da ein anderes Gas verwendet wird, werden auch andere Anregungsenergien erwartet.
Bei der Anregung von Neon durch Elektronen ist ein Übergang von Grundzustand in einen der 10 3p Zustände am wahrscheinlichsten. Diese Liegen in einem Bereich von 18,4 - 19 V \cite{HND1}.
Beim Übergang vom 3p in den 3s Zustand werden Photonen im sichtbaren teil des Spektrums emittiert. Der Übergang aus dem 3s Zustand in den Grundzustand emittiert Photonen im UV-Bereich.
\subsection{Heizen des Versuchsaufbaus}
Heizen des Versuchsaufbaus ergibt hier wenig Sinn, da nun ein Gas anstelle eines Dampfes vorliegt. Damit ändert sich die Teilchendichte nicht mit der Temperatur.
\section{Beobachtungen und Interpretation}
Die in \ref{tab:gemesseneAnregungsenergieNeon} gelisteten, gemessenen Maxima bestätigen die geäusserten Vermutungen.
Im Stoßraum können zudem pro gemessenen Peak eine leuchtende Scheibe gesehen werden. Diese Scheiben sind die Stoßzonen der verschiedenen Elektronen.
\begin{table}
	\centering
	\caption{Anregungsniveaus von Neon}
	\begin{tabular}{c c}
		Anzahl der Stöße & Gegenspannung des Peaks in V \\
		\midrule
		1 & 18,3 \\
		2 & 25,6 \\
		3 & 55,7 \\
	\end{tabular}
	\label{tab:gemesseneAnregungsenergieNeon}
\end{table}
Eine Lineare Regression der gemessenen Werte ergibt eine Steigung von 18,7 V / Anregung was in dem von der Vorbereitungsliteratur angegebenen Intervall liegt.


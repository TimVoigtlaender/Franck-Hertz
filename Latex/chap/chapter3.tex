In diesem Versuch ermitteln wir die Stoffe einiger unbekannter Präparate.
\begin{figure}
	\centering
	\includegraphics[width=\textwidth]{../Daten/Aufgabe 3/Alle_Messungen_2.png}
	\caption{Röntgenpeaks der gemessenen Präparate}
\end{figure}
Hierzu Eichen wir zunächst die Energieskala mithilfe des Bariumpeaks. Dieser hält sich bei Kanal 89,279 auf und sollte nach der in der Versuchsbeschreibung gegebenen Tabelle bei 32,1 keV liegen. Wir erhalten somit die Eichung:
\begin{align*}
E_{Kanal}=\frac{32,1\; keV}{89,279}\cdot n_{Kanal}
\end{align*}
Zur überprüfung dieses Wertes vergleichen wir auch den Bleipeak mit dem gegebenen Wert. Der gemessene Wert beträgt 72,486 keV und der literaturwert 74,1 keV. Wir erhalten hier alo einen Fehler von 2,18\%. Für das unbekannte Präparat B lesen wir 57,056 keV ab, was am ehesten zu Tantal passt, welches mit 57,1 keV angegeben ist. Für das unbekannte Präparat C lesen wir 58,682 keV ab, was am ehesten zu Wolfram mit 58,8 keV passt. Beim unbekannten Präparat D haben wir zusätzlich zum gesuchten Peak noch einen weiteren Peak gemessen, der wohl vom Barium stammt, mit dem das präparat bestrahlt wurde. Wir messen hier 65,904 keV, was am ehesten Platin (66,2 keV) entspricht. Beim unbekannte Präparat E ist es schwer einen Peak auszumachen. Beim Vergleich mit den anderen Graphen können wir jedoch eine kleine Erhöhung bei 22,897 keV feststellen. Dies entspricht etwa dem gegebenen Wert für Silber (22,1 keV). Die unbekannten Materialien sind also Tantal, Wolfram, Platin und Silber.